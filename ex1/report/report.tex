\documentclass[11pt]{article}

\usepackage{geometry}
\usepackage{amsthm}
\usepackage{amsmath}
\usepackage[utf8]{inputenc}
\usepackage{amsfonts}
\usepackage{lstfiracode}
\usepackage{fontspec}

\setmonofont{Fira Code}[
  Contextuals=Alternate
]
\newfontfamily{\firaseries}{Fira Code}
\usepackage{listings}
\usepackage{lstfiracode}
\lstset{
  language=Octave,
  basicstyle=\footnotesize\firaseries,
  numbers=none,
  showstringspaces=false,
  style=FiraCodeStyle,
  tabsize=2
}

\makeatletter
\lstdefinestyle{inline-style}{
  basicstyle=%
    \ttfamily
    \lst@ifdisplaystyle\scriptsize\fi
}
\makeatother

\newcommand*{\noaccsupp}[1]{\BeginAccSupp{ActualText={}}#1\EndAccSupp{}}

\theoremstyle{definition}
\newtheorem{definition}{Definition}[section]
\newtheorem{example}[definition]{Example}
\newtheorem{problem}[definition]{Problem}
\newtheorem{claim}[definition]{Claim}
\newtheorem{observation}[definition]{Observation}
\newtheorem{notation}[definition]{Notation}

\begin{document}

\title{Problem Solving By Computer - Project 1}
\date{}
\maketitle

%%%%%%%%%%%%%%%%%%%%%%%%%%%%%%%%%%%%%%%%%%%%%%%%%%%%%%%%%%%%
% Cubic Taxicab Number
%%%%%%%%%%%%%%%%%%%%%%%%%%%%%%%%%%%%%%%%%%%%%%%%%%%%%%%%%%%%
\section{Cubic Taxicab Number}
\begin{definition}
	A \textbf{cubic taxicab number} is a positive integer that can be expressed as the sum of two positive cubic numbers in at least two distinct ways.
\end{definition}

\begin{example}
	The first cubic taxicab number is $1729$, since $1729 = 1^3 + 12^3 = 9^3 = 10^3$.
\end{example}

\begin{problem}
	Given a positive integer $N$, how can we determine the smallest cubic taxicab number greater or equal to $N$? In particular, how can we write a function \lstinline|CubicTaxicabNum(N)| such that \lstinline|ctn = CubicTaxicabNum(N)| returns the smallest cubic taxicab number greater or equal to $N$?
\end{problem}

%%%%%%%%%%%%%%%%%%%%%%%%%%%%%%%%%%%%%%%%%%%%%%%%%%%%%%%%%%%%
% Approach
%%%%%%%%%%%%%%%%%%%%%%%%%%%%%%%%%%%%%%%%%%%%%%%%%%%%%%%%%%%%

\subsection{Approach}
Suppose that we have a function \lstinline|IsCubicTaxicabNum(N)| that determines whether a positive integer $N$ is a cubic taxicab number, then we are able to deduce the smallest cubic taxicab number greater or equal to $N$ by checking the integers greater than or equal to $N$ in turn, until we find a cubic taxicab number.\\

\begin{lstlisting}
function ctn = CubicTaxicabNum(N)
% CUBICTAXICABNUM    Returns the smallest cubic taxicab number greater
%                    than or equal to N

	ctn = N;

	while (~IsCubicTaxicabNum(ctn))
    	ctn = ctn + 1;
	end
end
\end{lstlisting}

\noindent
All that remains to do is to implement the function \lstinline|IsCubicTaxicabNum(n)|, such that the function returns $1$ if $n$ is a taxicab number and $0$ otherwise.

%%%%%%%%%%%%%%%%%%%%%%%%%%%%%%%%%%%%%%%%%%%%%%%%%%%%%%%%%%%%
% Is it a Cubic Taxicab Number?
%%%%%%%%%%%%%%%%%%%%%%%%%%%%%%%%%%%%%%%%%%%%%%%%%%%%%%%%%%%%

\subsection{Is it a Cubic Taxicab Number?}

In this section, we will look at how to implement the function \lstinline|IsCubicTaxicabNum(N)| that determines whether a positive integer $N$ is a cubic taxicab number. First, we will make some observations that will help us derive an implementation for \lstinline|IsCubicTaxicabNum(N)|.

\begin{claim}
	Suppose that $t$ is a positive integer such that $t = x^3 + y^3$, where $x, y$ are positive integers, then $x, y \leq \lfloor \sqrt[3]{t} \rfloor$.
\end{claim}
\begin{proof}
	Let $x, y, t$ be positive integers such that $t = x^3 + y^3$. Suppose by contradiction, we have $x > \lfloor \sqrt[3]{t} \rfloor$. Note that $\lceil \sqrt[3]{t} \rceil \geq \sqrt[3]{t}$. As $x$ is an integer, we must have $x \geq \lceil \sqrt[3]{t} \rceil$. Together, we have $x \geq \lceil \sqrt[3]{t} \rceil \geq \sqrt[3]{t}$, which implies $x^3 \geq \lceil \sqrt[3]{t} \rceil^3 \geq t$. In particular, for $t = x^3 + y^3$ to hold, we must have $t = x^3$, but this means that we must have $y = 0$, contradicting the choice of $y$ being a positive integer.
\end{proof}

\begin{observation}
	Suppose that $t$ is a positive integer such that $t = x^3 + y^3$, where $x, y$ are positive integers, then there is no positive integer $a \neq x$, such that $t = a^3 + y^3$. Similarly, there is no positive integer $b \neq y$, such that $t = x^3 + b^3$.
\end{observation}
\begin{proof}
	Let $a, x, y, t$ be positive integers such that $x^3 + y^3 = t$ and $a^3 + y^3 = t$. We have $x^3 + y^3 = t = a^3 + y^3$, and so $x^3 = a^3$. As both $x$ and $a$ are positive integers, we must have $x = a$. By symmetry, the latter statement also holds.
\end{proof}

\begin{observation}
	Suppose that $t$ is a positive integer such that $x^3 + y^3 < t$, where $x, y$ are positive integers, with $x \leq y$. Then, there is no positive integer $z < y$, with $x^3 + z^3 = t$.
\end{observation}
\begin{proof}
	Let $x, y, t$ be positive integers with $x \leq y$ and $x^3 + y^3 < t$. Let $z$ be any positive integer with $z < y$, then we must have $z^3 < y^3$. It follows that $x^3 + z^3 < x^3 + y^3 < t$ and so the result holds.
\end{proof}

\begin{observation}
	Suppose that $t$ is a positive integer such that $x^3 + y^3 > t$, where $x, y$ are positive integers, with $x \leq y$. Then, there is no integer $z > x$, with $z^3 + y^3 = t$.
\end{observation}
\begin{proof}
	Let $x, y, t$ be positive integers with $x \leq y$ and $x^3 + y^3 > t$. Let $z$ be any positive integer with $z > x$, then we must have $z^3 > x^3$. It follows that $z^3 + y^3 > x^3 + y^3 > t$ and so the result holds.
\end{proof}

\subsubsection{Algorithm}
As a result of the above observations, we have an algorithm for determining whether a positive integer $N$ is a cubic taxicab number. The pseudocode for the algorithm is given below.

\begin{lstlisting}
Algorithm IsCubicTaxicabNum(N)
	Let n = 0 be the number of distinct pairs we've found
	Let left = 1
	Let right = the integral part of the cube root of N
	
	While (left <= right)
		t = left^3 + right^3
		If (t = N)
			n = n + 1
			If (n = 2)
				Return Yes, as we have found two distinct pairs
			End If
			left = left + 1
			right = right - 1
		Else If (t < N)
			left = left + 1
		Else
			right = right - 1
		End If
	End While
	
	Return No, as we have not have found 2 distinct pairs
End IsCubicTaxicabNum
\end{lstlisting}

\subsubsection{Implementation}
The implementation of \lstinline|IsCubicTaxicabNum(N)| in MATLAB is as follows.

\begin{lstlisting}
function isTaxicab = IsCubicTaxicabNum(N)
% ISCUBICTAXICABNUM    Returns isTaxicab = 1, if N is a taxicab number
%                      Returns isTaxicab = 0, otherwise

nPairs = 0;

left = 1;
right = floor(nthroot(N, 3));

while (left < right)
    sum = left^3 + right^3;
    if (sum == N)
        nPairs = nPairs + 1;
        if (nPairs == 2)
            isTaxicab = true;
            return;
        end
        left = left + 1;
        right = right - 1;
    elseif (sum < N)
        left = left + 1;
    else
        right = right - 1;
    end
end

isTaxicab = false;

end
\end{lstlisting}

%%%%%%%%%%%%%%%%%%%%%%%%%%%%%%%%%%%%%%%%%%%%%%%%%%%%%%%%%%%%
% Analysis
%%%%%%%%%%%%%%%%%%%%%%%%%%%%%%%%%%%%%%%%%%%%%%%%%%%%%%%%%%%%

\subsection{Analysis}

\begin{example}
	The smallest cubic taxicab number greater or equal to 36032 is 39312. Indeed, 39312 is a cubic taxicab number since $39312 = 2^3 + 34^3 = 15^3 + 33^3$.
\end{example}

\newpage
%%%%%%%%%%%%%%%%%%%%%%%%%%%%%%%%%%%%%%%%%%%%%%%%%%%%%%%%%%%%
% Catalan's Constant
%%%%%%%%%%%%%%%%%%%%%%%%%%%%%%%%%%%%%%%%%%%%%%%%%%%%%%%%%%%%

\section{Catalan's Constant}

\begin{definition}
	The \textbf{Catalan's constant} is a mathematical constant named after Eug\'ene Charles Catalan, and is defined as
	$$G = \sum_{k=0}^\infty \frac{(-1)^k}{(2k+1)^2} = \frac{1}{1^2} - \frac{1}{3^2} + \frac{1}{5^2} + \dots \approx 0.915965594177219$$
	Although the Catalan's constant $G$ can be expressed in terms of the above sum of series, it is not known whether $G$ is irrational or not.
\end{definition}

\begin{problem}
	In many situations, it would be more convenient to approximate the Catalan's constant as a ratio of two positive integers. Given a positive integer $N$, what is the best rational approximation $p / q$ of the Catalan's constant, subject to the constraint $p + q \leq N$. In particular, how can we write a function \lstinline|RatAppCat(N)| such that \lstinline|[p, q] = RatAppCat(N)| returns the pair of integers $p$ and $q$ for the best rational approximation $p / q$ of $G$, such that $p + q \leq N$?
\end{problem}

%%%%%%%%%%%%%%%%%%%%%%%%%%%%%%%%%%%%%%%%%%%%%%%%%%%%%%%%%%%%
% Approach
%%%%%%%%%%%%%%%%%%%%%%%%%%%%%%%%%%%%%%%%%%%%%%%%%%%%%%%%%%%%

\subsection{Approximating Catalan's Constant}
\begin{observation}
	Given $G = 0.915965594177219$ and an integer $q$, we can find $p$ such that $p / q = G$, by computing $p \approx Gq$. Of course, $p$ needs not to be an integer; however, we can find the integer $p_0$ such that $p_0 / q$ is closest to $G$ among all integers by $p_0 = \text{round}(p)$.
\end{observation}

\begin{claim}
	Suppose we have integers $p_0, p_1, q_0, q_1$ such that $p_0 / q_0 = G$ and $p_1 / q_1 = G$. If we have $q_1 > q_0$, then it must the case $p_1 > p_0$.
\end{claim}

\begin{proof}
	Let $p_0, p_1, q_0, q_1$ be integers such that $p_0 / q_0 = G$, $p_1 / q_1 = G$ and $q_0 < q_1$. We have $p_0 = Gq_0$ and $p_1 = Gq_1$, but as $q_0 < q_1$, we have $Gq_0 < Gq_1$. It follows that $p_0 < p_1$.
\end{proof}

\subsubsection{Approach}
As a result of Observation 2.3, we can enumerate over values of $q$ in ascending order, computing the \textit{perfect} integer $p$ for each denominator $q$, and checking whether  $p / q$ is a better approximation than any previous pairs of $(p, q)$ considered. Further, when we encounter a pair $(p, q)$ such that $p + q > N$, then we are done. Indeed, as we are enumerating over values of $q$ in ascending order, any subsequent pairs we consider, say $(p_0, q_0)$, we will have $q_0 > q_1$, and so $p_0 > p_1$ by Claim 2.4. In particular, $p_0 + q_0 > p + q > N$.

\newpage
%%%%%%%%%%%%%%%%%%%%%%%%%%%%%%%%%%%%%%%%%%%%%%%%%%%%%%%%%%%%
% Implementation
%%%%%%%%%%%%%%%%%%%%%%%%%%%%%%%%%%%%%%%%%%%%%%%%%%%%%%%%%%%%

\subsection{Implementation}
The implementation of \lstinline|RatAppCat(N)| in MATLAB is as follows.

\begin{lstlisting}
function [p, q] = RatAppCat(N)
% RATAPPCAT    Returns the best approximation p / q of the Catalan's 
%              constant, among all pairs of (p, q) such that p + q <= N.

G = 0.915965594177219;
p = 0; q = 1;                          % The best (p, q) pair so far
minDelta = abs(G - p / q);             % The difference between p / q and G

for q0 = 1 : N
  p0 = round(G * q0);                  % Observation 2.3
  if (p0 + q0 > N)                     % Claim 2.4
    return
  end
  delta = abs(G - p0 / q0);              
  if (delta < minDelta)                % Update if current pair is better
    minDelta = delta;                  % than the best pair we have seen
    p = p0; q = q0;
  end
end
end
\end{lstlisting}

%%%%%%%%%%%%%%%%%%%%%%%%%%%%%%%%%%%%%%%%%%%%%%%%%%%%%%%%%%%%
% Analysis
%%%%%%%%%%%%%%%%%%%%%%%%%%%%%%%%%%%%%%%%%%%%%%%%%%%%%%%%%%%%

\subsection{Analysis}

\begin{example}
	For $N = 2018$, the pair of integers $109$ and $119$ provides the best rational approximation of the Catalan's constant, with $109/119 = 0.915966386554622$.
\end{example}

\subsubsection{Complexity}
The implementation of \lstinline|RatAppCat(N)| loops over possible denominators from $1$ to $N$, and for each possible denominator, we compute the perfect numerator, and check whether this pair gives a better approximation than any previously encountered pairs. The operations in each iteration are run in constant time, and so the time complexity of \lstinline|RatAppCat(N)| is $O(N)$, as the loop is run at most $N$ times. The space complexity of \lstinline|RatAppCat(N)| is $O(1)$, as we only store a constant number of variables.

















\newpage
%%%%%%%%%%%%%%%%%%%%%%%%%%%%%%%%%%%%%%%%%%%%%%%%%%%%%%%%%%%%
% Sum of Reciprocal Squares with Prime Factors
%%%%%%%%%%%%%%%%%%%%%%%%%%%%%%%%%%%%%%%%%%%%%%%%%%%%%%%%%%%%

\section{Sum of Reciprocal Squares with Prime Factors}

\begin{definition}
	The \textbf{sum of reciprocal squares}, also known as the Basel problem, asks for the precise summation of the reciprocal squares of the natural numbers. That is,
	$$\sum_{n=1}^\infty \; \frac{1}{n^2} = \frac{1}{1^2} + \frac{1}{2^2} + \frac{1}{3^2} + \frac{1}{4^2} + \dots$$
	which was shown to be exactly $\pi^2/6$ by Leonhard Euler.
\end{definition}

\begin{definition}
	The \textbf{prime omega function (with multiplicity)} $\Omega(n)$ counts the number of prime factors of a natural number $n$, counting multiplicity. By convention, we have $\Omega(1) = 0$.
\end{definition}

\begin{example}
	$\Omega(12) = 3$ as $12 = 2^2 \cdot 3^1$, $\Omega(25) = 2$ as $25 = 5^2$ and $\Omega(47) = 1$ as $47$ is a prime number. Moreover, $\Omega(p) = 1$ for any prime number $p$, as prime numbers only have one prime factor, by definition.
\end{example}

\begin{definition}
	The \textbf{sum of reciprocal squares with prime factors} asks for the precise summation of the following sum.
	$$\sum_{n=1}^\infty \; \frac{(-1)^{\Omega(n)}}{n^2} = \frac{1}{1^2} - \frac{1}{2^2} - \frac{1}{3^2} + \frac{1}{4^2} + \dots$$
\end{definition}

\begin{problem}
	How can we derive a reasonable approximation of the sum of the reciprocal squares with prime factors? In particular, can we make any observations about the answer by truncating a finite number of terms?
\end{problem}

%%%%%%%%%%%%%%%%%%%%%%%%%%%%%%%%%%%%%%%%%%%%%%%%%%%%%%%%%%%%
% Approach
%%%%%%%%%%%%%%%%%%%%%%%%%%%%%%%%%%%%%%%%%%%%%%%%%%%%%%%%%%%%

\subsection{Approximating the Sum of Reciprocal Squares with Prime Factors}

\begin{observation}
	For any $a, b \in \mathbb{N}$ with $a < b$, we have
	$$-\sum_{n=a}^b \frac{1}{n^2} \leq \sum_{n=a}^b \frac{(-1)^{\Omega(n)}}{n^2} \leq \sum_{n=a}^b \frac{1}{n^2} \quad \text{and} \quad -\sum_{n=a}^\infty \frac{1}{n^2} \leq \sum_{n=a}^\infty \frac{(-1)^{\Omega(n)}}{n^2} \leq \sum_{n=a}^\infty \frac{1}{n^2}$$
\end{observation}

\begin{observation}
	Let $x, y, z \in \mathbb{R}$ with $x \leq y \leq z$. Let $n \in \mathbb{N} \cup \{0\}$. Let \lstinline|round(a, n)| be a function that rounds $a$ to $n$ decimal places. If \lstinline|round(x, n) = round(z, n) = c|, then \lstinline|round(y, n) = c|. That is, if $x$ and $z$ round to the same value for $n$ decimal places, say $c$, then $y$ rounded to $n$ decimal places is also $c$.
\end{observation}

\begin{notation}
$$S(a, b) = \sum_{n = a}^b \frac{(-1)^{\Omega(n)}}{n^2} \quad \text{and} \quad T(a, b) = \sum_{n = a}^b \frac{1}{n^2}$$
\end{notation}
\noindent
We know that $T(1, \infty) = \pi^2 / 6$ and our goal is to approximate $S(1, \infty)$.

\newpage
\begin{observation}
	We can compute $T(k + 1, \infty)$ by $T(k + 1, \infty) = T(1, \infty) - T(1, k) = \pi^2 / 6 - T(1, k)$. That is, we can compute $T(k + 1, \infty)$ by subtracting the first $k$ terms from $\pi^2 / 6$.
\end{observation}

\subsubsection{Approach}

Let $k \in \mathbb{N}$. It follows from Observation 3.6 that we have
$$S(1, k) - T(k + 1, \infty) \leq \underbrace{S(1, k) + S(k + 1, \infty)}_{S(1, \infty)} \leq S(1, k) + T(k + 1, \infty)$$

\noindent
In particular, for some $d \in \mathbb{N}$, if we are able to find $k$ such that the above inequality holds, with the lower and upper bound rounding to the same value (up to $d$ decimal places), then by Observation 3.7, this value is an accurate approximation for $S(1, \infty)$ up to $d$ decimal places.

%%%%%%%%%%%%%%%%%%%%%%%%%%%%%%%%%%%%%%%%%%%%%%%%%%%%%%%%%%%%
% Implementation
%%%%%%%%%%%%%%%%%%%%%%%%%%%%%%%%%%%%%%%%%%%%%%%%%%%%%%%%%%%%

\subsection{Implementation}
\subsubsection{Prime Omega Function}
First, we need to implement the function \lstinline|PrimeOmega(N)| such that \lstinline|n = PrimeOmega(N)| returns the number of prime factors of $N$ (counting multiplicity). In fact, we can make use of the \lstinline|factor(N)| function provided by MATLAB, which returns a vector containing the prime factors of $N$. It follows that \lstinline|PrimeOmega(N)| can be implemented by counting the length of the vector returned by \lstinline|factor(N)| and can be implemented as an anonymous function, as follows.

\begin{lstlisting}
PrimeOmega = @(x) (x != 1) * length(factor(x));
\end{lstlisting}

\noindent
As \lstinline|factor(1) = [1]|, we use \lstinline|(x != 1)| to handle the case $\Omega(1) = 0$. That is, if $x \neq 1$, then \lstinline|PrimeOmega(x)| returns \lstinline|length(factor(x))|, otherwise it returns \lstinline|0|.

\subsubsection{Finding the value of $k$}
Let $d$ be the number of decimal places which we wish to get an approximation for $S(1, \infty)$. Let $k = 0$, $S = 0$ and $T = \pi^2 / 6$. We can enumerate over values of $k$ in turn, adding $(-1)^{\Omega(k)} / k^2$ to $S$, and subtracting $1 / k^2$ from $T$. If $S - T$ and $S + T$ round to the same value up to $d$ decimal places, then we have an approximation for $S(1, \infty)$ up to $d$ decimal places, by considering the first $k$ terms.

\newpage
\subsubsection{Sum of Reciprocal Squares with Prime Factors}
The implementation of \lstinline|SumPF| in MATLAB is as follows.

\begin{lstlisting}
function SumPF
% SUMPF    Finds an approximation of the sum of reciprocal squares
%          with prime factors

S = 0;
T = pi^2 / 6;
k = 0; d = 1;
PrimeOmega = @(n) (n != 1) * length(factor(n));

fprintf('%-8s %-8s %8s\n', 'Terms', 'Value', 'Accuracy');
while (k < 1000000)
  if (round(S - T, d) == round(S + T, d))
    fprintf('%-8d %8f %-8d\n', k, round(S, d), d);
    d = d + 1;
  end
  k = k + 1;
  S = S + (-1)^PrimeOmega(k) / k^2;
  T = T - 1 / k^2;
end
end
\end{lstlisting}

%%%%%%%%%%%%%%%%%%%%%%%%%%%%%%%%%%%%%%%%%%%%%%%%%%%%%%%%%%%%
% Analysis
%%%%%%%%%%%%%%%%%%%%%%%%%%%%%%%%%%%%%%%%%%%%%%%%%%%%%%%%%%%%

\subsection{Analysis}

\begin{lstlisting}
>> SumPF
Terms    Value    Accuracy
123      0.700000 1       
334      0.660000 2       
2102     0.658000 3       
42353    0.658000 4       
728565   0.657970 5       
\end{lstlisting}

\begin{example}
	By considering the first $2102$ terms of the sum of reciprocal squares with prime factors, we get that $S = 0.6580$ to 4 decimal places. Similarly, we get that $S = 0.65797$ to 5 decimal places, by considering $728565$ terms.
\end{example}

\subsubsection{Alternate Approach}
Our implementation of \lstinline|SumPF| makes use of the sum of reciprocal squares, which was shown to be $\pi^2 / 6$. As a result, our implementation would not be suitable to compute an approximation for an arbitrary sum. Alternatively, we could derive an approach that deduces such bounds without knowledge of another summation.


\end{document}

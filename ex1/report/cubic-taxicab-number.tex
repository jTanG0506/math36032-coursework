%%%%%%%%%%%%%%%%%%%%%%%%%%%%%%%%%%%%%%%%%%%%%%%%%%%%%%%%%%%%
% Cubic Taxicab Number
%%%%%%%%%%%%%%%%%%%%%%%%%%%%%%%%%%%%%%%%%%%%%%%%%%%%%%%%%%%%
\section{Cubic Taxicab Number}
\begin{definition}
	A \textbf{cubic taxicab number} is a positive integer that can be expressed as the sum of two positive cubic numbers in at least two distinct ways.
\end{definition}

\begin{example}
	The first cubic taxicab number is $1729$, since $1729 = 1^3 + 12^3 = 9^3 = 10^3$.
\end{example}

\begin{problem}
	Given a positive integer $N$, how can we determine the smallest cubic taxicab number greater or equal to $N$? In particular, how can we write a function \lstinline|CubicTaxicabNum(N)| such that \lstinline|ctn = CubicTaxicabNum(N)| returns the smallest cubic taxicab number greater or equal to $N$?
\end{problem}

%%%%%%%%%%%%%%%%%%%%%%%%%%%%%%%%%%%%%%%%%%%%%%%%%%%%%%%%%%%%
% Approach
%%%%%%%%%%%%%%%%%%%%%%%%%%%%%%%%%%%%%%%%%%%%%%%%%%%%%%%%%%%%

\subsection{Approach}
Suppose that we have a function \lstinline|IsCubicTaxicabNum(N)| that determines whether a positive integer $N$ is a cubic taxicab number, then we are able to deduce the smallest cubic taxicab number greater or equal to $N$ by checking the integers greater than or equal to $N$ in turn, until we find a cubic taxicab number.\\

\lstinputlisting{../ex1/CubicTaxicabNum.m}

\noindent
All that remains to do is to implement the function \lstinline|IsCubicTaxicabNum(N)|, such that the function returns \lstinline|1| if $N$ is a taxicab number and \lstinline|0| otherwise.

%%%%%%%%%%%%%%%%%%%%%%%%%%%%%%%%%%%%%%%%%%%%%%%%%%%%%%%%%%%%
% Is it a Cubic Taxicab Number?
%%%%%%%%%%%%%%%%%%%%%%%%%%%%%%%%%%%%%%%%%%%%%%%%%%%%%%%%%%%%
\newpage
\subsection{Is it a Cubic Taxicab Number?}

Before looking at the implementation of the function \lstinline|IsCubicTaxicabNum(N)|, we will first make some observations that will help us derive an implementation for \lstinline|IsCubicTaxicabNum(N)|.

\begin{claim}
	Suppose that $t$ is a positive integer such that $t = x^3 + y^3$, where $x, y$ are positive integers, then $x, y \leq \lfloor \sqrt[3]{t} \rfloor$.
\end{claim}
\begin{proof}
	Let $x, y, t$ be positive integers such that $t = x^3 + y^3$. Suppose by contradiction, we have $x > \lfloor \sqrt[3]{t} \rfloor$. Note that $\lceil \sqrt[3]{t} \rceil \geq \sqrt[3]{t}$. As $x$ is an integer, we must have $x \geq \lceil \sqrt[3]{t} \rceil$. Together, we have $x \geq \lceil \sqrt[3]{t} \rceil \geq \sqrt[3]{t}$ and so $x^3 \geq \lceil \sqrt[3]{t} \rceil^3 \geq t$. In particular, for $t = x^3 + y^3$ to hold, we must have $t = x^3$, but this means that we must have $y = 0$, contradicting the choice of $y$ being a positive integer.
\end{proof}

\begin{observation}
	Suppose that $t$ is a positive integer such that $t = x^3 + y^3$, where $x, y$ are positive integers, then there is no positive integer $a \neq x$, such that $t = a^3 + y^3$. Similarly, there is no positive integer $b \neq y$, such that $t = x^3 + b^3$.
\end{observation}
\begin{proof}
	Let $a, x, y, t$ be positive integers such that $x^3 + y^3 = t$ and $a^3 + y^3 = t$. We have $x^3 + y^3 = t = a^3 + y^3$, and so $x^3 = a^3$. As both $x$ and $a$ are positive integers, we must have $x = a$. By symmetry, the latter statement also holds.
\end{proof}

\begin{observation}
	Suppose that $t$ is a positive integer such that $x^3 + y^3 < t$, where $x, y$ are positive integers, with $x \leq y$. Then, there is no positive integer $z < y$, with $x^3 + z^3 = t$.
\end{observation}
\begin{proof}
	Let $x, y, t$ be positive integers with $x \leq y$ and $x^3 + y^3 < t$. Let $z$ be any positive integer with $z < y$, then we must have $z^3 < y^3$. It follows that $x^3 + z^3 < x^3 + y^3 < t$ and so the result holds.
\end{proof}

\begin{observation}
	Suppose that $t$ is a positive integer such that $x^3 + y^3 > t$, where $x, y$ are positive integers, with $x \leq y$. Then, there is no integer $z > x$, with $z^3 + y^3 = t$.
\end{observation}
\begin{proof}
	Let $x, y, t$ be positive integers with $x \leq y$ and $x^3 + y^3 > t$. Let $z$ be any positive integer with $z > x$, then we must have $z^3 > x^3$. It follows that $z^3 + y^3 > x^3 + y^3 > t$ and so the result holds.
\end{proof}

\newpage
\subsubsection{Implementation}
The implementation of \lstinline|IsCubicTaxicabNum(N)| in MATLAB is as follows.

\begin{lstlisting}
function isTaxicab = IsCubicTaxicabNum(N)
% ISCUBICTAXICABNUM    Returns isTaxicab = 1, if N is a taxicab number
%                      Returns isTaxicab = 0, otherwise

nPairs = 0;
left = 1;
right = floor(nthroot(N, 3));                    % Claim 1.4
while (left <= right)
    sum = left^3 + right^3;
    if (sum == N)
        nPairs = nPairs + 1;
        if (nPairs == 2)                         % If we have found the
            isTaxicab = true;                    % second pair, we are done
            return;
        end
        left = left + 1;                         % Observation 1.5
        right = right - 1;                       % Observation 1.5
    elseif (sum < N)
        left = left + 1;                         % Observation 1.6
    else
        right = right - 1;                       % Observation 1.7
    end
end
isTaxicab = false;
end
\end{lstlisting}

%%%%%%%%%%%%%%%%%%%%%%%%%%%%%%%%%%%%%%%%%%%%%%%%%%%%%%%%%%%%
% Analysis
%%%%%%%%%%%%%%%%%%%%%%%%%%%%%%%%%%%%%%%%%%%%%%%%%%%%%%%%%%%%

\subsection{Analysis}

\begin{example}
	We can verify that the first cubic taxicab number is indeed 1729 by calling \lstinline|CubicTaxicabNum(1)|.
\end{example}

\begin{example}
	The smallest cubic taxicab number greater or equal to 36032 is 39312. Indeed, 39312 is a cubic taxicab number since $39312 = 2^3 + 34^3 = 15^3 + 33^3$.
\end{example}

\subsubsection{Complexity}
The implementation of \lstinline|IsCubicTaxicabNum(N)| makes use of the \textit{two-pointer technique}, in which we have two pointers, \lstinline|left| and \lstinline|right|, and we insist that \lstinline|left <= right|. We compute the sum of the cubes, \lstinline|S = left^3 + right^3| and if \lstinline|S < N|, then we increment \lstinline|left| by 1, and if \lstinline|S > N|, then we decrement \lstinline|right| by 1. If \lstinline|S == N|, then we increment \lstinline|left| by 1 and decrement \lstinline|right| by 1. It follows that we have at most $\sqrt[3]{N}$ iterations of the \lstinline|while| loop, and the operations inside the \lstinline|while| loop run in constant time, so the time complexity of \lstinline|IsCubicTaxicabNum(N)| is $O(\sqrt[3]{N})$.

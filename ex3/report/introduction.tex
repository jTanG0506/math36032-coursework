\section*{Introduction}
In the e-commerce world, there is an importance in being able to predict the behaviour of customers to survive in the market. In particular, is there a relation between a product being returned and the rating that was left by the customer? Moreover, are customers less likely to make future orders if they have previously returned a product for a refund? Finally, how can we determine the most valuable customers by deriving a ranking system which takes multiple attributes into account? \\

\noindent
In this project, we will analyse online transactions that were collected over four years for a given large online retailer. The data set is provided as a CSV (comma-separated values) file, and the first few lines of the data file are as follows. \\

\begin{lstlisting}
   Date      Customer_ID  Product_Category  Product_Value  Rating  Return
___________  ___________  ________________  _____________  ______  ______
01-Jan-2015    1010408           B              33.5         5       N 
01-Jan-2015    1014220           C              18.4         4       N 
01-Jan-2015    1016167           E              23.2         4       N 
01-Jan-2015    1019094           D                46         0       Y 
01-Jan-2015    1019535           D              25.5         0       N
\end{lstlisting}
$ $

\noindent
The schema of the data file is as follows.
\begin{lstlisting}[language=none]
Date: online transaction date (from 01 Jan 2015 to 31 Dec 2018)
Customer_ID: anonymised customer ID
Product_Category: one of 'A', 'B', 'C', 'D' or 'E'
Product_Value: the total amount of the order, in £GBP
Rating: the customer rating as a 1-5 star rating system, and 0 if no rating
Return: whether the product was returned for refund, N or Y
\end{lstlisting}

\noindent
We will start by reading the CSV file into a table. MATLAB uses type inference to determine the appropriate data types for the values in each column. In order to see the data types inferred by MATLAB, we can use the function \lstinline|summary(AllOrders)|.
\begin{lstlisting}
AllOrders = readtable('purchasing_order.csv');
summary(AllOrders)
\end{lstlisting}

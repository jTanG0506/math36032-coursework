\documentclass[11pt]{article}

\usepackage{geometry}
\usepackage{amsthm}
\usepackage{amsmath}
\usepackage[utf8]{inputenc}
\usepackage{amsfonts}
\usepackage{lstfiracode}
\usepackage{fontspec}
\usepackage{enumitem}

\setmonofont{Fira Code}[
  Contextuals=Alternate
]
\newfontfamily{\firaseries}{Fira Code}
\usepackage{listings}
\usepackage{lstfiracode}
\lstset{
  language=Octave,
  basicstyle=\footnotesize\firaseries,
  numbers=none,
  showstringspaces=false,
  style=FiraCodeStyle,
  tabsize=2
}

\lstdefinelanguage{none}{
  identifierstyle=
}

\makeatletter
\lstdefinestyle{inline-style}{
  basicstyle=%
    \ttfamily
    \lst@ifdisplaystyle\scriptsize\fi
}
\makeatother

\newcommand*{\noaccsupp}[1]{\BeginAccSupp{ActualText={}}#1\EndAccSupp{}}

\newcommand\diagram[1]{
	\begin{center}
		\makebox[\textwidth]{\includegraphics[width=0.8\textwidth]{figures/#1}}
	\end{center}
}

\theoremstyle{definition}
\newtheorem{definition}{Definition}[section]
\newtheorem{example}[definition]{Example}
\newtheorem{problem}[definition]{Problem}
\newtheorem{claim}[definition]{Claim}
\newtheorem{observation}[definition]{Observation}
\newtheorem{notation}[definition]{Notation}

\begin{document}

\title{Problem Solving By Computer - Project 3}
\date{}
% \author{Jonathan Tang}
\maketitle

\section*{Introduction}
In the e-commerce world, there is an importance in being able to predict the behaviour of customers to survive in the market. In particular, is there a relation between a product being returned and the rating that was left by the customer? Moreover, are customers less likely to make future orders if they have previously returned a product for a refund? Finally, how can we determine the most valuable customers by deriving a ranking system which takes multiple attributes into account? \\

\noindent
In this project, we will analyse online transactions that were collected over four years for a given large online retailer. The data set is provided as a CSV (comma-separated values) file, and the first few lines of the data file are as follows. \\

\begin{lstlisting}
   Date      Customer_ID  Product_Category  Product_Value  Rating  Return
___________  ___________  ________________  _____________  ______  ______
01-Jan-2015    1010408           B              33.5         5       N 
01-Jan-2015    1014220           C              18.4         4       N 
01-Jan-2015    1016167           E              23.2         4       N 
01-Jan-2015    1019094           D                46         0       Y 
01-Jan-2015    1019535           D              25.5         0       N
\end{lstlisting}
$ $

\noindent
The schema of the data file is as follows.
\begin{lstlisting}[language=none]
Date: online transaction date (from 01 Jan 2015 to 31 Dec 2018)
Customer_ID: anonymised customer ID
Product_Category: one of 'A', 'B', 'C', 'D' or 'E'
Product_Value: the total amount of the order, in £GBP
Rating: the customer rating as a 1-5 star rating system, and 0 if no rating
Return: whether the product was returned for refund, N or Y
\end{lstlisting}

\noindent
We will start by reading the CSV file into a table. MATLAB uses type inference to determine the appropriate data types for the values in each column. In order to see the data types inferred by MATLAB, we can use the function \lstinline|summary(AllOrders)|.
\begin{lstlisting}
AllOrders = readtable('purchasing_order.csv');
summary(AllOrders)
\end{lstlisting}

\newpage
\section{Return rate on low rating items}
Since the majority of the customers who returned an order gave a low rating, we will start by finding a relation between the probability $P(r)$ that a product is likely to be returned and the rating $r$ the customer gave, using the following logistic function with two parameters $\alpha$ and $\beta$.

$$P(r) = \frac{1}{1 + \exp(-\alpha r - \beta)}$$

\noindent
That is, if the rating is low, then the customer is more likely to return the product. In order to reduce the overall computation complexity, we will only consider the ratings of those customers who have returned a product for refund.

\subsection{Preprocessing the data}
As we wish to only consider the ratings of those customers who have returned at least one product for refund, we first need to filter out orders from the data set that do not satisfy these requirements. It follows that we can first obtain the list of customer IDs for those who have returned a product for refund by first filtering for orders that were returned, then using MATLAB's \lstinline|unique()| function to remove any duplicates.

\begin{lstlisting}
ReturnedOrders = AllOrders(strcmp(AllOrders.Return, 'Y'), :);
Users_With_Returns = unique(ReturnedOrders.Customer_ID);
\end{lstlisting}

\noindent
Next, we need to obtain the orders that were placed by customers that have returned at least one order, and have a rating left by the customer. That is, we need to obtain the set of orders whose \lstinline|Customer_ID| belongs in the \lstinline|Users_With_Returns| list, and have \lstinline|Rating > 0|. We can obtain such subset of orders by using logical indexing as follows.

\begin{lstlisting}
Orders = AllOrders(ismember(AllOrders.Customer_ID, Users_With_Returns), :);
OrdersWithRatings = Orders(Orders.Rating > 0, :);
\end{lstlisting}

\noindent
That is, \lstinline|OrdersWithRating| contains orders from \lstinline|AllOrders| such that the customers who placed the orders have returned at least one product for refund.

\subsection{Finding the parameters $\alpha$ and $\beta$}
Suppose we have a function $f(\alpha, \beta)$ which tells us the lack of fit of the logistic function $P(r, \alpha, \beta)$, then we can minimise the lack of fit by using MATLAB's \lstinline|fminsearch()| function, which finds the minimum of a multi-variable function. That is, we can create an error function $f(\alpha, \beta)$ which tells us the lack of fit of the logisitic function $P(r)$ with parameters $\alpha$ and $\beta$, then use \lstinline|fminsearch()| to minimise the error.

\subsubsection{Models for the error function}
We will use the non-linear least squares model for our error function. As the name suggests, the error is given by the sum of the square of the errors. The non-linear least squares error function is as follows.

$$f(\alpha, \beta) = \sum_i \; \Big(p_i - P(r_i)\Big)^2 = \sum_i \; \Bigg(p_i - \frac{1}{1 + \exp(-\alpha r_i - \beta)}\Bigg)^2$$

\noindent
where $(r_i, p_i)$ is a transaction with rating $r_i \in [1, ..., 5]$ and a boolean flag $p_i \in \{0, 1\}$ which indicates whether the order was refunded, with 1 indicating that the order was refunded. \\

\noindent
The implementation of the non-linear least squares error function is as follows.

\begin{lstlisting}
function S = errorFunction(a, r, p)
  S = 0;
  for k = 1 : length(r)
    S = S + (p(k) - 1 / (1 + exp(-a(1) * r(k) - a(2))))^2;
  end
end
\end{lstlisting}

\noindent
It follows that we can use find the parameters $\alpha$ and $\beta$ by minimising the error function.

\begin{lstlisting}
orderRating = OrdersWithRatings.Rating;
orderIsReturned = strcmp(OrdersWithRatings.Return, 'Y');
lrParams = fminsearch(@(a) errorFunction(a, orderRating, orderIsReturned), [0 0]);
fprintf('α = %.5f, β = %.5f\n', lrParams); 
\end{lstlisting}

\subsection{Results}

We conclude that the relation between the probability $P(r)$ that a product is likely to returned and the rating $r$ the customer gave, is governed by the following logistic function.

$$P(r) = \frac{1}{1 + \exp(17.542411 r - 17.418797)}$$

\noindent
That is, we conclude that the parameters are $\alpha = -17.542411$ and $\beta = 17.418797$. Moreover, we can plot the logistic function against the data points as follows.

\begin{lstlisting}
hx = linspace(0, 6, 1001);
plot(orderRating, isOrderReturned, 'o', ...
     hx, 1 ./ (1 + exp(-lrParams(1) * hx - lrParams(2))));
lg = legend('Raw Data', 'Logistic Regression');
set(lg,'Location', 'east');
xlabel('Rating'); xticks([1 : 5]);
ylabel('Order Returned'); yticks([0, 1]); yticklabels({'No', 'Yes'});
axis([0.5, 5.5, -0.1, 1.1]);
\end{lstlisting}

\diagram{logistic_function}

% TODO: Conclusion on the parameters and what they mean


\end{document}

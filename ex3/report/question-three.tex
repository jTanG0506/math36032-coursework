\newpage
\section{Determining the most valuable customers}
In e-commerce, the customer lifetime value (CLV) is the value a customer contributes to your business over their lifetime at your company. For example, we could say that customer who purchases many products from Category C are valuable, as these products have the highest profit margin. Similarly, we could say that customers that leave high ratings on our products are valuable, as they are more likely to recommend our products through word of mouth. \\

\noindent
For simplicity, all customers will be ranked according to the following criteria:
\begin{enumerate}[label=(\arabic*)]
  \item average product value per order in category C
  \item the average rating left on orders
\end{enumerate}

\noindent
Moreover, we wish to design our ranking system in such a way that customers ranked top according to one criteria are still likely to be on top in the ranking with both criteria. That is, the two given criteria are of equal importance and should be weighted equally. For example, a customer with a high average product value per order in Category C that has left no ratings on orders, should still be towards the top in our ranking system.

\subsection{Calculating the average order value and rating}
In order to get the average product value per order in Category C by customer, we start by filtering for orders with product category C, then use \lstinline|groupsummary()| to group records by customer ID and take the mean of the product values. The implementation is as follows.

\begin{lstlisting}
OrdersFromC = Orders(strcmp(Orders.Product_Category, 'C'), :);
CatCValue = groupsummary(OrdersFromC, {'Customer_ID'}, 'mean', 'Product_Value');
\end{lstlisting}

\noindent
Similarly, we can get the average rating for each customer by using the \lstinline|groupsummary()| method on the subset of orders which have a rating. Given that the a rating of 0 implies that a particular product was not rated by the customer, we can get the subset of orders which have a rating by simply checking whether the rating is non-zero.

\begin{lstlisting}
OrdersWithRatings = Orders(Orders.Rating > 0, :);
Ratings = groupsummary(OrdersWithRatings, {'Customer_ID'}, 'mean', 'Rating');
\end{lstlisting}

\newpage
\subsection{Creating a customer table using outer join}

% TODO: Description of why we wish to create such customer table.
% TODO: Diagram showing how outer join of two tables works.

\begin{lstlisting}
TLeft = table(CatCValue.Customer_ID, CatCValue.mean_Product_Value, ...
  'VariableNames', {'Customer_ID', 'Mean_Value'});
TRight = table(Ratings.Customer_ID, Ratings.mean_Rating, ...
   'VariableNames', {'Customer_ID', 'Mean_Rating'});
T = outerjoin(TLeft, TRight, 'MergeKeys', true);
\end{lstlisting}

\noindent
Observe that customers that have either not placed any orders for products in category C or not left any ratings on orders will have a \lstinline|NaN| value in the table \lstinline|T|. Further, any customer who has never placed an order for a product in category C and has never left a rating on a order will not appear in \lstinline|T|. However, that is the expected behaviour, as we cannot rank customers who have no orders that satisfy the ranking criteria. We will replace any potential \lstinline|NaN| values with 0 using the \lstinline|fillmissing()| function.

\begin{lstlisting}
T = fillmissing(T, 'constant', 0);
\end{lstlisting}

\noindent
All that remains is to derive a ranking based on the values \lstinline|Mean_Value| and \lstinline|Mean_Rating|. \\

\subsection{How do we rank the customers?}
We have a table \lstinline|T| whose rows consists of customer IDs along with their average order value for products in category C and the average rating they left across all orders. The first few rows of the table \lstinline|T| are as follows.

\begin{lstlisting}
                Customer_ID      Mean_Value        Mean_Rating   
                ___________    ______________    _______________
                  1010269              47.575      4.77777777778
                  1010289               25.74      4.83333333333
                  1010302               18.44      4.71428571429
                  1010314               40.72                4.8
                  1010320        33.566666667                4.8  
\end{lstlisting}

\noindent
We can gain insights regarding the range of the data in our table by calling \lstinline|summary(T)|.

\begin{lstlisting}
      Mean_Value: 913x1 double              Mean_Rating: 913x1 double            
        Values:                               Values:            
          Min                0                  Min       2.444444444                
          Median         31.75                  Median    4.363636363                
          Max             82.3                  Max                 5    
\end{lstlisting}

\noindent
Observe that the range of values for the average order value is much greater than the range of values for the average rating. If we was to take the mean of the two values, the rating criteria would favour a higher average rating. In order to ensure that the two attributes have equal weighting, we normalise the columns respectively.

\begin{lstlisting}
T = normalize(T, 'norm', Inf, 'DataVariables', {'Mean_Value', 'Mean_Rating'});
Ranking = table(T.Mean_Value + T.Mean_Rating, 'VariableNames', {'Ranking'});
T = [T Ranking];
\end{lstlisting}

\subsection{Results}

In order to get the most valuable customers, we need to sort the table \lstinline|T| based on the \lstinline|Ranking| column, in descending order. That is, a customer with a higher ranking is a more valuable customer, and perhaps we may wish to reward the most valuable customers with a coupon for their next purchase as a sign of gratitude.

\begin{lstlisting}
T = sortrows(T, 'Ranking', 'Descend');
disp(T(1:10, {'Customer_ID', 'Ranking'}));
\end{lstlisting}

\noindent
The first and last few customers ID's of the top 100 customers are listed below with their respective ranking score.

\begin{lstlisting}
                    Rank    Customer_ID        Ranking                                    
                    ____    ___________    ________________                               

                       1      1011981      1.96194036825872                               
                       2      1014288       1.8757080100944                               
                       3      1014953      1.86153846153846                               
                       4      1012195      1.81470230862697                               
                       5      1014622      1.78196300796544                               
                       6      1016309      1.77545565006075                               
                       7      1016326       1.7579181855002                               
                       8      1015292      1.74448079259744                               
                       9      1011918      1.72833805859322                               
                      10      1010983      1.71454029971648                               
                                ...              ...                                      
                                ...              ...                                      
                                ...              ...                                      
                      90      1016337      1.51968408262454                               
                      91      1015223      1.51756692072609                               
                      92      1014365      1.51545344084834                               
                      93      1018272      1.51533193416547                               
                      94      1011746      1.51381862366066                               
                      95      1011044      1.51246456055083                               
                      96      1016208      1.51063735778195                               
                      97      1012212      1.50794653705954                               
                      98      1016515      1.50784652771287                               
                      99      1013090      1.50660186310247                               
                     100      1019307      1.50495968187341                               
\end{lstlisting}

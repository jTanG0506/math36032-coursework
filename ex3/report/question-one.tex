\newpage
\section{The return rate on low rating items}
Since the majority of the customers who returned an order gave a low rating, we will start by finding a relation between the probability $P(r)$ that a product is likely to be returned and the rating $r$ the customer gave, using the following logistic function with two parameters $\alpha$ and $\beta$.

$$P(r) = \frac{1}{1 + \exp(-\alpha r - \beta)}$$

\noindent
That is, if the rating is low, then the customer is more likely to return the product. To reduce the overall computation complexity, we will only consider the ratings of those customers who have returned a product for a refund.

\subsection{Preprocessing the data}
As we wish only to consider the ratings of those customers who have returned at least one product for a refund, we first need to filter out orders from the data set that do not satisfy this requirement. First, we can obtain the list of customer IDs for those customers who have returned a product for a refund by filtering for orders that were returned, followed by MATLAB's \lstinline|unique()| function to remove any duplicates.

\begin{lstlisting}
ReturnedOrders = AllOrders(strcmp(AllOrders.Return, 'Y'), :);
UsersWithReturns = unique(ReturnedOrders.Customer_ID);
\end{lstlisting}

\noindent
Next, we need to obtain the orders that were placed by customers that have returned at least one order and have a rating left by the customer. That is, we need to obtain the set of orders whose \lstinline|Customer_ID| belongs in the \lstinline|UsersWithReturns| list, and have \lstinline|Rating > 0|. We can obtain such orders by using logical indexing as follows.

\begin{lstlisting}
Orders = AllOrders(ismember(AllOrders.Customer_ID, UsersWithReturns), :);
OrdersWithRatings = Orders(Orders.Rating > 0, :);
\end{lstlisting}

\noindent
That is, \lstinline|OrdersWithRating| contains orders from \lstinline|AllOrders| that have a rating, and the customers who placed the orders have returned at least one product for a refund.

\subsection{Finding the parameters $\alpha$ and $\beta$}
Suppose we have a function $f(\alpha, \beta)$ which tells us the lack of fit of the logistic function $P(r, \alpha, \beta)$, then we can minimise the lack of fit by using MATLAB's \lstinline|fminsearch()| function, which finds the minimum of a multi-variable function. That is, we can create an cost function $f(\alpha, \beta)$ which tells us the lack of fit of the logisitic function $P(r)$ with parameters $\alpha$ and $\beta$, then use \lstinline|fminsearch()| to find the values of $\alpha$ and $\beta$ which correspond to the minimum cost.

\subsubsection{Models for the cost function}
We will use the non-linear least squares model for our cost function. As the name suggests, the error is given by the sum of the square of the errors. The non-linear least squares error function is as follows.

$$f(\alpha, \beta) = \sum_i \; \Big(p_i - P(r_i)\Big)^2 = \sum_i \; \Bigg(p_i - \frac{1}{1 + \exp(-\alpha r_i - \beta)}\Bigg)^2$$

\noindent
where $(r_i, p_i)$ is an order with rating $r_i \in [1, ..., 5]$ and a boolean flag $p_i \in \{0, 1\}$ which indicates whether the order was refunded, with 1 indicating that the order was refunded.

\noindent
The implementation of the non-linear least squares cost function is as follows.

\begin{lstlisting}
function S = costFunction(a, r, p)
  S = 0;
  for k = 1 : length(r)
    S = S + (p(k) - 1 / (1 + exp(-a(1) * r(k) - a(2))))^2;
  end
end
\end{lstlisting}

\noindent
Finally, we use \lstinline|fminsearch| to find $\alpha$ and $\beta$ by minimising the error function, as follows.

\begin{lstlisting}
orderRating = OrdersWithRatings.Rating;
isReturned = strcmp(OrdersWithRatings.Return, 'Y');
lrParams = fminsearch(@(a) costFunction(a, orderRating, isReturned), [0 0]);
fprintf('[alpha, beta] = [%.5f, %.5f]\n', lrParams); 
\end{lstlisting}

\subsection{Results}

We conclude that the relation between the probability $P(r)$ that a product is  returned and the rating $r$ the customer gave, is governed by the following logistic function.

$$P(r) = \frac{1}{1 + \exp(17.542411 r - 17.418797)}$$

\noindent
That is, we conclude that the parameters are $\alpha = -17.542411$ and $\beta = 17.418797$. Further, we can plot the logistic function against the data points as follows.

\begin{lstlisting}
hx = linspace(0, 6, 1001);
plot(orderRating, isReturned, 'o', ...
     hx, 1 ./ (1 + exp(-lrParams(1) * hx - lrParams(2))));
lg = legend('Raw Data', 'Logistic Regression');
set(lg, 'Location', 'east');
xlabel('Rating'); xticks([1 : 5]);
ylabel('Order Returned'); yticks([0, 1]); yticklabels({'No', 'Yes'});
axis([0.5, 5.5, -0.1, 1.1]);
\end{lstlisting}

\diagram{logistic_function}

% TODO: Conclusion on the parameters and what they mean
